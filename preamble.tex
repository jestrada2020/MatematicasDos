header-includes:
- \usepackage[utf8]{inputenc} #para usar accentos
- \usepackage{graphicx}
- \usepackage[labelfont=bf]{caption}
- \usepackage{float}
- \usepackage{afterpage} # agrega opcion de insertar página en blanco
## Formato de titulo de capitulo
- \usepackage{titlesec, blindtext, color}
- \usetikzlibrary{calendar}
- \definecolor{gray75}{gray}{0.75}  # define un estilo de color
- \newcommand{\hsp}{\hspace{20pt}}  # define un espacio fijo
# Define el formato de capitulo 
- \titleformat{\chapter}[hang]{\huge\bfseries}{\textcolor{gray75}{$\blacksquare$}\hsp}{0pt}{\huge\bfseries\MakeUppercase}
- \setcounter{secnumdepth}{0} #quita la numeración de capitulos
#- \setlength{\parindent}{1.5em} # indentacion de párrafos en la primera linea
- \addtolength{\headheight}{3pt}
## Formato encabezado y pie de paginas:
- \usepackage{fancyhdr} # para insertar encabezados y pies de pag personalizados
- \pagestyle{fancy} # establece estilo personalizado de encabezados
- \renewcommand{\chaptermark}[1]{\markboth{#1}{}} # para quitar num cap en leftmark
#- \renewcommand{\sectionmark}[1]{\markright{\arabic{section}.\#1}} # para quitar num cap en leftmark
- \renewcommand{\headrulewidth}{0.5pt} # añade línea de arriba en encabezado
- \renewcommand{\footrulewidth}{0.5pt} # linea abajo en pie de pag
#     thesection= num capitulo
#     leftmark= nombre capitulo completo
#     rightmark = nombre subcapitulo
- \lhead{\scriptsize \rightmark} # marca de subcapitulo en la parte izq arriba de encabezado
#- \chead{} # zona centro encabezado
- \rhead{\scriptsize \leftmark} 
#- \lfoot{\textit{www.gritos.org}} # zona izq pie pag
#- \cfoot{\includegraphics[height=0.08\textwidth]{imag/logo1.jpg}} # zona centro pie pag incluye imagen
#- \rfoot{\thepage}
## Formato nombres genericos de listas automáticas
## No activar babel pues coge los del idioma entonces
- \AtBeginDocument{\renewcommand{\contentsname}{Índice}}
- \AtBeginDocument{\renewcommand{\listfigurename}{Lista de figuras}}
- \AtBeginDocument{\renewcommand{\listtablename}{Lista de Tablas}}
- \AtBeginDocument{\renewcommand{\figurename}{Fig. }}
- \AtBeginDocument{\renewcommand{\tablename}{Tabla. }}
- \AtBeginDocument{\renewcommand{\chaptername }{Capítulo }}
\usepackage{booktabs}
\usepackage{amsmath,amssymb,amsfonts,latexsym,cancel}
\usepackage{rawfonts}
\usepackage{pictexwd}
\usepackage{tikz}
	\usetikzlibrary{shapes,arrows,spy,positioning,snakes}
